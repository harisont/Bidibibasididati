% !TeX spellcheck = IT
\documentclass[12pt]{article}
\usepackage[utf8]{inputenc}
\usepackage[italian]{babel}
\usepackage{graphicx}
\usepackage{float} %per posizione assoluta delle immagini [H]
\usepackage[colorlinks=true,linkcolor=blue]{hyperref}
\usepackage{nameref} 
\graphicspath{{./images/}}
\def\code#1{\texttt{#1}}
\def\image[#1][#2]#3{
  \begin{figure}[H]
  \centering
  \includegraphics[#2]{#1}
  \caption{#3}
  \end{figure}}

\title{Il database sottostante Git}
\author{Arianna Masciolini}

\begin{document}
\maketitle
\newpage
\tableofcontents
\newpage
\section{Introduzione ai VCS}
Un \textit{Version Control System} (VCS) è un sistema che tiene traccia delle modifiche apportate ad uno o più file in modo da garantire all'utente la possibilità di accedere alle versioni precedenti dei file suddetti in qualsiasi momento. 
\bigskip \\
I primi sistemi di controllo di versione, locali, nacquero con l'idea di risolvere i problemi legati a quello che potrebbe essere definito versioning "manuale", consistente nel conservare più copie dei file d'interesse: la forte suscettibilità a errori e lo spreco di spazio su disco. Tra questi, RCS ha goduto ha lungo di grande popolarità: esso salva su disco, in un particolare formato, una serie di \textit{patch}, ossia le differenze tra una versione e l'altra dei file, in modo tale da poter ricostruire lo stato in cui era ognuno di essi in qualsiasi momento, applicandovi una dopo l'altra le varie patch.
\bigskip \\
Successivamente, ci si pose il problema di permettere a più persone di collaborare a distanza. Per risolverlo nacquero i sistemi centralizzati di controllo di versione (CVCS), come CVS, Subversion, e Perforce. In questi sistemi, per il resto analoghi ad RCS e simili, tutte le versioni dei file controllati sono salvate su un unico server e rese così disponibili ai diversi utenti. Anche questo approccio presenta però problematiche importanti, dovute al fatto che il server centrale rappresenta un punto di vulnerabilità per l'intero sistema. 
\bigskip \\
I DVCS (\textit{Distributed VCS}), di cui Git, Mercurial, Bazaar e Darcs sono gli esempi più noti, risolvono questo problema: i membri del gruppo non si limitano a scaricare la più recente versione dei file, ma copiano l'intero repository, cosicchè ogni client costituisce un backup completo del progetto.
\image[lcd.png][scale=0.8]{Confronto tra un VCS locale, un CVCS e un DVCS}
\newpage
\begin{thebibliography}{9}
	\bibitem{git_pro}
	Scott Chacon, Ben Straub. \textit{Git Pro}, seconda edizione, 2014.
\end{thebibliography}
\end{document}